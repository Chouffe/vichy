\documentclass{llncs}

\usepackage[utf8]{inputenc}

%\usepackage{graphicx}
%
\begin{document}


\title{Vichy: the collaborative editor}
\subtitle{Get ready to love it, resistance is futile\ldots}

\author{Arthur Caillau, Caroline Keramsi, Paul Lagrée, Bertrand Mermet, Romain Pomier}

\institute{School of Computer Science and Communication, Royal Institute of Technology KTH, Stockholm}
\maketitle

\begin{abstract}

The purpose of this project is to allow several persons to modify the same document at the same time, the changes being applied on every versions. The behavior is then the same as the one of a Google Doc. Instead of using collaborative editing in a browser, this article presents how we made it a plugin for the text editor vim. Different modules and technologies have been used, among them the NoSQL database MongoDB which is used to store all the data of users and the modifications they do. We use these data to have some statistics about the developping process. For instance, we can know who made the last change. Beside this, ShareJS, an Operational Transform library for NodeJS, helps to deal with concurrent editing and to synchronise all the differents versions available. 

\end{abstract}

\section{Introduction}\label{sec:Introduction}

Introduction is here.

\section{Other Sections}\label{sec:Others}

Other sections are here. 


\section{Conclusion}\label{sec:Conclusion}

Conclusions are here.

\section*{Acknowledgments}\label{sec:Acknowledgments}

Authors would like to thank YYYYY.

\begin{thebibliography}{1}

\bibitem{Einstein}
A. Einstein, On the movement of small particles suspended in stationary liquids required by the molecular-kinetic theory of heat, Annalen der Physik 17, pp. 549-560, 1905.

\end{thebibliography}

\end{document}
