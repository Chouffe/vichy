\documentclass{llncs}

\usepackage[utf8]{inputenc}
\usepackage{hyperref}
%\usepackage{minted}

%\usepackage{graphicx}
%
\begin{document}


\title{Vichy: the collaborative editor}
\subtitle{Get ready to love it, resistance is futile\ldots}

\author{Romain Pomier,\\Arthur Caillau, Caroline Keramsi, Paul Lagrée, Bertrand Mermet}

\institute{School of Computer Science and Communication, Royal Institute of Technology KTH, Stockholm}
\maketitle

\begin{abstract}

The purpose of this project is to allow several persons to modify the same
document at the same time, the changes being applied on every versions.
The behavior is then the same as the one of a Google Doc.
Instead of using collaborative editing in a browser, this article presents a
solution to give the text editor Vim a plugin for collaborative editing.
Different modules and technologies have been used, among them the NoSQL
database MongoDB which is used to store all the data of users and the
modifications they do.
We use these data to have some statistics about the developing process.
For instance, we can know who made the last change. Besides this, ShareJS, an
Operational Transform library for NodeJS, is used to deal with concurrent
editing and to synchronise all the different versions available.

\end{abstract}

\section{Introduction}\label{sec:Introduction}

The development of the Internet has changed the way people work.
Whereas one used to share carefully the work among members of a team between
two meetings, one now work at the same time on the same parts of the project
thanks to real-time tools.
The success of the famous Google Documents is a good example of the development
of collaborative work for the general public.
The idea of this project was to adapt the same concept to Vim, a well-known
text editor very popular among nerds.
//TODO CHANGE
This paper presents in the first part the method which has been used on the
client side to extract the modifications in Vim.
In the second part, we explain the database structure and how we managed to
solve the problem of concurrency with ShareJS.
Some statistics are extracted from the database to give every user some
information about the coding process.
END TODO CHANGE //
Finally, we explain in the conclusion how the solution can be improved or at
least made differently.

\section{Server-side}\label{sec:Server-side}

Allowing users to collaboratively edit documents is not an easy task.
What happens if two users modify the document at the same time ?
They could even modify the same word ! A class of algorithms was developed to handle these difficulties:
the Operational Transform algorithms. Given two concurrent operations, these kind
of algorithms are able to decide which must be applied first and how the second
operation must be transformed and applied in order to achieve a consistent result.
Developing a complete Operational Transform system for this project seemed unrealistic.
Instead we decided to use an open-source library called ShareJS. 
ShareJS provides a way to embed a collaborative editor in a web page.
The library is made of two parts: one on the client-side in JavaScript, 
and one on the server-side which comes as a module for the JavaScript server-side framework
Node.js.
ShareJS does not provide directly all the features that we need. 
We had to add the possibility to undo an operation and to authenticate an user in order to add
history and management of users rights.

ShareJS is a young project, not yet stable and well-documented.
Bertrand corrected a small bug in the connection of ShareJS with MongoDB

\subsection{Database}
ShareJS can be used with many databases as back-end. 
The available databases are: Redis, CouchDB, MongoDB, PostgreSQL and MySQL.
We chose to use the NoSQL database MongoDB for several reasons. As a starting
point we were curious to discover the trending world of NoSQL databases. So we
first did some research to make sure that NoSQL was suitable for our use that we
defined as the abilities to:

\begin{itemize}
        \item Store a current snapshot of the document as well as some metadata.
        \item Keep a record of all operations performed on a document so that
            they can be easily processed to extract statistics and allow us to
            add an undo operation on top of ShareJS.
        \item Store information about users and rights on document to allow
            secure editing through authentication and fine user rights
            configuration.
\end{itemize}

MongoDB satisfies all these needs. It is able to represent documents, operations,
and user information as JSON structured document. They are then organized in
collections similar to the RDBMS concept of tables but that do not enforce a schema.
This property actually happened to be very useful later on the project since it
allowed us to add some additional fields in the existing ShareJS database without
breaking it. Furthermore MongoDB provides powerful aggregation features through
its aggregation framework and map-reduce implementation that we used to compute
statistics on the database (see listing~\ref{query_example} for an example).

%\begin{listing}
%    \begin{minted}[frame=single,
%            framesep=3mm,
%            xleftmargin=21pt,
%        tabsize=4]{js}
%        db.ops.aggregate(
%            {$match : {
%                "_id.doc": docName }
%            },
%            {$group :{
%                _id: "$opData.meta.name",
%                ops: {$sum : 1}}
%            }
%        );
%    \end{minted}
%    \caption{Query computing total number of users operations on a document} 
%    \label{query_example}
%\end{listing}
%\subsection{Database structure}

Documents collections:
%\begin{listing}
%    \begin{minted}[frame=single,
%            framesep=3mm,
%            xleftmargin=21pt,
%        tabsize=4]{js}
%        {
%            "_id": string,
%            "data": {
%                "v": number,
%                "meta": {
%                    "creator": string,
%                    "mtime": number,
%                    "ctime": number,
%                    "name": string
%                },
%                "snapshot": string,
%                "type": string
%            }
%        }
%    \end{minted}
%    \caption{Documents collection schema} 
%    \label{docs_collection}
%\end{listing}
%
%\begin{listing}
%    \begin{minted}[frame=single,
%            framesep=3mm,
%            xleftmargin=21pt,
%        tabsize=4]{js}
%        {
%            "opData": {
%                "op": [objects],
%                "meta": {
%                    "source": string,
%                    "name": string,
%                    "ts": number
%                }
%            },
%            "_id": {
%                "doc": string,
%                "v": number
%            }
%        }
%    \end{minted}
%    \caption{Operations collection schema} 
%    \label{ops_collection}
%\end{listing}

\subsection{Management of the documents}

We created a website to allow the users to manage their documents.
To build quickly the building system, we used the template Node-login which is built
on Node.js and MongoDB.
Once the user is logged, he can create documents and then share them with the other
users existing on the website. The user can also open his documents inside the browser.

\section{Implementation of the undo functionality}\label{sec:Others}

\section{Extraction of Vim modifications}\label{sec:Others}

\subsection{Use of the NetBeans Protocol}

The NetBeans protocol of Vim is a text based communication protocol over a
classical TCP socket.
It permits any environment providing a socket interface to control Vim using
this protocol.\cite{netbeans} Using this protocol, we can then create a client
that will be a link between our server who is merging the different version of
the document, and the Vim instance.
The client will be notified every time the user modifies the document, and will
update the buffer in Vim when it receives modifications from the server.
It would have been possible to have vim directly connected to the distant
server, but then the server would have had the full power over vim clients,
which is not good for obvious security reasons.\\
We used an implementation of the NetBeans Protocol for Node.js called
\textit{node-vim-netbeans}.\cite{node-vim-netbeans}

\subsection{Vim plugin}

To show in Vim some additional information about the shared document, we
created a plugin written in \textit{Vim script} and importing python functions
which are simpler to write.\\
We implemented a \textit{blame} function opening a new buffer next to the
current one, and showing who was the last person to modify each line.

\section{Conclusion}\label{sec:Conclusion}

In the end, the collaborative editing in Vim works with our implementation.
It has been quite difficult to choose the right method to implement our collaborative plugin
because Vim gives few tools to access the modifications a user does. With more time,
we could have improved the visualization of the statistics to show them in a more friendly way.

\begin{thebibliography}{1}

\bibitem{netbeans}
Vim documentation defining this protocol: \url{http://vimdoc.sourceforge.net/htmldoc/netbeans.html}
\bibitem{node-vim-netbeans}
node-vim-netbeans on github: \url{https://github.com/clehner/node-vim-netbeans}

\end{thebibliography}

\end{document}
